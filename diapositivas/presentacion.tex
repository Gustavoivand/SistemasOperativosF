\documentclass{beamer}
\usetheme{Madrid}
\usecolortheme[named=blue]{structure}

% Paquetes necesarios
\usepackage{graphicx}
\usepackage{listings}
\usepackage{booktabs}
\usepackage{multicol}
\usepackage[spanish]{babel}
\usepackage[utf8]{inputenc}
\usepackage{caption}
\usepackage{tikz}
\usepackage{bookmark}

% Pie de página con número
\setbeamertemplate{footline}[page number]

% Configuración de listings para código
\lstset{
  language=Python,
  basicstyle=\ttfamily\footnotesize,
  keywordstyle=\color{blue},
  commentstyle=\color{gray},
  stringstyle=\color{orange},
  showstringspaces=false,
  breaklines=true,
  frame=single,
  literate={á}{{\'a}}1 {é}{{\'e}}1 {í}{{\'i}}1 {ó}{{\'o}}1 {ú}{{\'u}}1 {ñ}{{\~n}}1
}
\graphicspath{ {imagenes/} }

% Datos de portada
\title{Control Inteligente de Ambientes con Tecnología IoT}
\subtitle{Sistema Automatizado usando MicroPython y FreeRTOS}
\author{Alex Vega, Jean Paul Mori, Gustavo Delgado}
\institute{Universidad Nacional de Ingeniería}
\date{\today}

% Comando para notas del presentador
\newcommand{\presenternote}[1]{\footnotesize{\textcolor{gray}{[Nota: #1]}}}

\begin{document}

% Portada
\begin{frame}
  \titlepage
  %\presenternote{Saludo inicial: "Buenos días/tardes, hoy presentaremos un sistema inteligente que monitorea y controla ambientes automáticamente"}
\end{frame}

% Índice
\begin{frame}{Qué veremos hoy}
  \tableofcontents[hideallsubsections]
  %\presenternote{Explicar brevemente cada sección: "Primero entenderemos el sistema, luego cómo funciona internamente, veremos resultados prácticos y finalmente conclusiones"}
\end{frame}

% 1. Resumen
\section{Resumen}
\begin{frame}{Resumen en 1 minuto}
  \begin{itemize}
    \item Sistema inteligente que monitorea temperatura y humedad
    \item Control automático de ventiladores y luces
    \item Funciona con tecnología similar a los dispositivos domésticos inteligentes
    \item Sistema que ``piensa'' por sí mismo usando MicroPython
    \pause
    \item \textcolor{orange}{Resultados:} Funcionamiento estable y eficiente
  \end{itemize}
  
  \vspace{5mm}
  \begin{tikzpicture}
    \draw[orange, fill=orange!20] (0,0) rectangle (0.98\textwidth,0.5);
    \node at (0.49\textwidth,0.25) {\footnotesize\textbf{Analogía}: Como un mayordomo digital que vigila y ajusta el ambiente};
  \end{tikzpicture}
  
  %\presenternote{Comparar con termostatos inteligentes: "Es similar a un Nest o sistema domótico pero más especializado"}
\end{frame}

% 2. Introducción
\section{Introducción}
\begin{frame}{Motivación y Objetivos}
  \begin{block}{Problema a resolver}
    \begin{itemize}
      \item Mantener condiciones ambientales óptimas
      \item Reducir consumo energético
      \item Evitar supervisión humana constante
    \end{itemize}
  \end{block}
  \pause
  \begin{exampleblock}{Nuestra solución}
    \begin{itemize}
      \item Dispositivo compacto con sensores
      \item Toma decisiones automáticas
      \item Se comunica con otros sistemas
    \end{itemize}
  \end{exampleblock}
  
  \presenternote{Ejemplo práctico: ``En invernaderos, fábricas o incluso hogares donde necesitas mantener condiciones específicas''}
\end{frame}

% 3. Estado del Arte
\section{Tecnologías clave}
\begin{frame}{Tecnologías clave explicadas}
  \begin{columns}
    \column{0.6\textwidth}
    \begin{itemize}
      \item \textbf{ESP32}: El ``cerebro'' del sistema (como un mini-computador)
      \item \textbf{MicroPython}: Lenguaje para programar el dispositivo
      \item \textbf{FreeRTOS}: Sistema que gestiona tareas simultáneas
      \item \textbf{MQTT}: Protocolo de comunicación (como WhatsApp para dispositivos)
    \end{itemize}
    \column{0.4\textwidth}
    \includegraphics[width=\textwidth]{f01_ESP32.jpg}
  \end{columns}
  
  \vspace{5mm}
  \begin{tikzpicture}
    \draw[blue, fill=blue!10] (0,0) rectangle (0.98\textwidth,0.5);
    \node at (0.49\textwidth,0.25) {\footnotesize\textbf{FreeRTOS} = Jefe que coordina múltiples trabajadores};
  \end{tikzpicture}
  
  \presenternote{``FreeRTOS es como el director de orquesta que coordina todos los instrumentos''}
\end{frame}

% 4. Arquitectura y Desarrollo
\section{Cómo funciona el sistema}
\begin{frame}{Componentes principales}
  \begin{columns}
    \column{0.5\textwidth}
      \begin{itemize}
        \item \textcolor{green}{Sensores}: Ojos que miden temperatura/humedad
        \item \textcolor{red}{Actuadores}: Manos que controlan luces/ventiladores
        \item \textcolor{blue}{Cerebro}: Procesa información y toma decisiones
        \item \textcolor{purple}{Comunicación}: Envía reportes a otros sistemas
      \end{itemize}
    \column{0.5\textwidth}
      \includegraphics[width=0.9\textwidth]{7.VisualizaciondeDatos.png}
      \captionof{figure}{Visualización de datos en tiempo real}
  \end{columns}
  
  \presenternote{``Pueden ver en la imagen cómo se muestran los datos que recoge el sistema''}
\end{frame}

\begin{frame}{Gestión inteligente de tareas}
  \begin{columns}
    \column{0.6\textwidth}
    \textbf{Cómo maneja múltiples tareas:}
    \begin{itemize}
      \item Monitoreo constante de sensores
      \item Control de actuadores según necesidades
      \item Comunicación con otros sistemas
      \item Todo simultáneamente sin colapsar
    \end{itemize}
    \column{0.4\textwidth}
    \includegraphics[width=\textwidth]{7.VisualizaciondeDatos.png}
  \end{columns}
  
  \vspace{3mm}
  \begin{alertblock}{Clave del sistema}
    FreeRTOS permite hacer múltiples tareas a la vez como un malabarista experto
  \end{alertblock}
  
  \presenternote{``Imaginen un chef que cocina, vigila el horno y atiende pedidos al mismo tiempo - así funciona nuestro sistema''}
\end{frame}

% 5. Código Clave
\section{El lenguaje del sistema}
\begin{frame}[fragile]{Cómo ``piensa'' el dispositivo}
\begin{lstlisting}[language=Python]
# Configuración de un LED inteligente
class LED:
    def __init__(self, pin_num, freq=1500):
        self.pwm = PWM(Pin(pin_num), freq=freq)
    
    def on(self, percentage=0):
        duty = int(percentage / 100 * 1023)
        self.pwm.duty(duty)  # Ajusta brillo
\end{lstlisting}

\vspace{3mm}
\begin{tikzpicture}
  \draw[green, fill=green!10] (0,0) rectangle (0.98\textwidth,0.4);
  \node at (0.49\textwidth,0.2) {\footnotesize\textbf{Explicación}: El sistema ajusta luces automáticamente según necesidades};
\end{tikzpicture}

\presenternote{``Este código es como las instrucciones que le damos al sistema para controlar las luces eficientemente''}
\end{frame}

\begin{frame}[fragile]{Toma de decisiones automática}
\begin{lstlisting}[language=Python]
def push_data():
    # Envia datos de sensores cada 10 segundos
    ...
    
while True:
    push_data()  # Reporta datos
    check_messages()  # Recibe instrucciones
    adjust_environment()  # Ajusta condiciones
    time.sleep(0.1)  # Espera breve
\end{lstlisting}

\vspace{3mm}
\begin{tikzpicture}
  \draw[orange, fill=orange!10] (0,0) rectangle (0.98\textwidth,0.4);
  \node at (0.49\textwidth,0.2) {\footnotesize\textbf{Ciclo básico}: Medir → Decidir → Actuar → Reportar};
\end{tikzpicture}

\presenternote{``Como nuestro ciclo diario: despertar, trabajar, comer, dormir... pero para dispositivos''}
\end{frame}

% 6. Comparativa de Rendimiento
\section{Rendimiento del sistema}
\begin{frame}{Eficiencia comparada}
  \begin{table}[]
    \centering
    \begin{tabular}{@{}lcc@{}}
      \toprule
      \textbf{Característica} & \textbf{Nuestro sistema} & \textbf{Sistemas tradicionales} \\
      \midrule
      Tiempo respuesta & 15 ms & 50-100 ms \\
      Mensajes por segundo & 40 & 10-20 \\
      Consumo energía & Bajo & Moderado \\
      \bottomrule
    \end{tabular}
    \caption{Comparativa de rendimiento}
  \end{table}
  
  \vspace{3mm}
  \begin{tikzpicture}
    \draw[blue, fill=blue!10] (0,0) rectangle (0.98\textwidth,0.4);
    \node at (0.49\textwidth,0.2) {\footnotesize\textbf{Interpretación}: Responde más rápido y maneja más información};
  \end{tikzpicture}
  
  \presenternote{``15ms es más rápido que un parpadeo humano (100-300ms)''}
\end{frame}

% 7. Resultados Experimentales
\section{Resultados prácticos}
\begin{frame}{Demostración visual}
  \begin{columns}
    \column{0.5\textwidth}
      \begin{figure}
        \includegraphics[width=0.75\textwidth]{11.ConexionWifi.png}
        \caption{Conexión estable a red}
      \end{figure}
      \vspace{0.2cm}
      \begin{figure}
        \includegraphics[width=0.75\textwidth]{12.PruebasDiversas.png}
        \caption{Pruebas de funcionamiento}
      \end{figure}
    \column{0.5\textwidth}
      \begin{figure}
        \includegraphics[width=0.75\textwidth]{13.PruebasFinaales.png}
        \caption{Operación continua 24/7}
      \end{figure}
      \vspace{0.2cm}
      \begin{figure}
        \includegraphics[width=0.55\textwidth]{7.VisualizaciondeDatos.png}
        \caption{Interfaz de monitoreo}
      \end{figure}
  \end{columns}
  
  \presenternote{``En estas imágenes podemos ver el sistema en acción durante nuestras pruebas''}
\end{frame}

% 8. Conclusiones y Futuro
\section{Conclusiones}
\begin{frame}{Lo que logramos}
  \begin{itemize}
    \item Sistema autónomo que mantiene condiciones ideales
    \item Respuesta rápida a cambios ambientales
    \item Comunicación efectiva con otros dispositivos
    \item Uso eficiente de energía
  \end{itemize}
  
  \pause
  \vspace{5mm}
  \begin{exampleblock}{Beneficios prácticos}
    \begin{itemize}
      \item Ahorro energético hasta 30%
      \item Reducción de supervisión humana
      \item Prevención de daños por condiciones adversas
    \end{itemize}
  \end{exampleblock}
  
  \presenternote{``En un invernadero, esto podría prevenir pérdidas por heladas o calor excesivo automáticamente''}
\end{frame}

\begin{frame}{Futuras mejoras}
  \begin{columns}
    \column{0.6\textwidth}
    \begin{itemize}
      \item \textcolor{green}{Integración con apps móviles}
      \item \textcolor{blue}{Reconocimiento de patrones}
      \item \textcolor{red}{Control por voz}
      \item \textcolor{purple}{Mayor eficiencia energética}
    \end{itemize}
    \column{0.4\textwidth}
    \includegraphics[width=\textwidth]{11.ConexionWifi.png}
  \end{columns}
  
  \vspace{5mm}
  \presenternote{``Imaginen controlar todo con un 'Hola dispositivo, ajusta la temperatura a 22°C'''}
\end{frame}

\begin{frame}{Gracias}
  \centering
  \Huge{¿Preguntas?}
  
  \vspace{5mm}
  \large{Contacto:}
  
  \vspace{5mm}
  \includegraphics[width=0.3\textwidth]{uni.png}
  
  \presenternote{Preparar respuestas para preguntas comunes sobre costos, aplicaciones prácticas y escalabilidad}
\end{frame}

\end{document}